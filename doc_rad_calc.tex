\documentclass[12pt, letterpaper]{article}
\usepackage[utf8]{inputenc}
\usepackage[pdftex]{graphicx}
\usepackage{xspace}
\usepackage{hyperref}
\usepackage{fullpage}
\usepackage{subcaption}
\usepackage{placeins}
\hypersetup{
    colorlinks=true,
    linkcolor=blue,
    filecolor=magenta,      
    urlcolor=cyan,
}
\graphicspath{ {/home/tobiasjenegger/Documents/summary/} }
\title{Radius/Momentum Calculation for S444 Experiment February 2020 - Overview}
\author{Tobias Jenegger}
\date{}
\begin{document}
\begin{titlepage}
\maketitle
\end{titlepage}
\subsection{The Setup}
\includegraphics[width=1.0\textwidth]{mes_wixh.png}
\includegraphics[width=1.0\textwidth]{SETUP_around_Target.png}
\section{Geometry and relative position of the detectors in the beam direction}
Here, the positions are given for the s444 and s467 experiments\\
\\
z position of the MWPC0: \hspace{10mm} zMW0 = -2520 mm\\
z position of the target: \hspace{10mm} zT = -684.5 mm\\
z position of the MWPC1 in front of the Twin-MUSIC:  \hspace{10mm}   zM1 = 279 mm \\
z position of the middle of the Twin-MUSIC:  \hspace{10mm}  zTwin = 553 mm\\
z position of the MWPC2 after the Twin-MUSIC:  \hspace{10mm}   zM2 = 854 mm\\
$\alpha$ tilted angle of GLAD (14 degrees):   \hspace{10mm} = 0.244 rad\\
effective length of GLAD:  \hspace{10mm}   L\textunderscore eff = 2067 mm\\
z middle of GLAD  \hspace{10mm}  zGm = 2577mm\\
horizontal of the central path (18 degree) \hspace{10mm}   $\theta$\textunderscore out0 = pi/10 rad\\
z position of the MWPC3 after GLAD   \hspace{10mm}  zM3 = 5937 mm\\
z position of the ToFwall          \hspace{10mm}     zToFW = 6660.2 mm\\
\\
Correspondence between the GLAD current and the magnetic field:  I = 3584 A, B = 2.2 T \\
\\
Positions of the TOFWPads:\\
1 $\Rightarrow$ Messel\\
27$\Rightarrow$ Wixhausen \\

\section{RUNS used for calibration = SWEEP RUNS without target}

\begin{center}
	\includegraphics[width=1.0\textwidth]{runs_screenshot.png}
\end{center}

\subsubsection{Other RUNS used for various checks:}
RUN 70: 2 cm C target\\
RUN 80: 10.86 mm C target\\
RUN 81: 24.53 mm CH2 target\\
RUN 67: 24 mm CH2 target \\
RUN 68: 1 cm C target \\
RUN 79: 12.29 mm CH2 target\\
RUN 75: 21.98mm C target\\
\\

\section{Methods for flightpath reconstruction in the (x,z) plane}
\subsection{The "Kickplane" method}
\subsubsection{From MW0 to the entrance of GLAD, the ion is following a straight line}\label{sssec:num1}
\begin{center}
	\includegraphics[width=1.0\textwidth]{tracking_upstreamGLAD.png}
\end{center}
\textbf{The straight line trajectory from MW0 to  entrance before glad is defined by:}\\
\textbf{$\Rightarrow$ one absolute value before GLAD}\\
absolute = calibrated position in mm in the laboratory frame\\
To get the position, use the position given by one MWPCs (1 or 2)\\
\\
\textbf{$\Rightarrow$ the theta angle (theta\textunderscore in) before GLAD}\\
Angle obtained from comining MWPC1 and MWPC2\\
(to get higher precision the drift time in TWIM MUSIC could be used)\\
\subsubsection{From entrance to the exit of GLAD, the effective trajectory is circular}
\begin{center}
	\includegraphics[width=1.0\textwidth]{tracking_GLAD.png}
\end{center}
\textbf{The circular trajectory is defined by:\\
$\Rightarrow$ one absolute position before GLAD B and angle theta\textunderscore in (see: \ref{sssec:num1})\\
$\Rightarrow$ one absolute position at MWPC3}\\
\\
From this information the angle theta\textunderscore out is constructed in follwing steps:\\
\begin{enumerate}
	\item Extend the line of flight of the ion before the GLAD.
	\item The point of intersection with the "kickplane" (symmetry axis line of GLAD magnet) is the kickpoint C.
	\item Draw a straight line between C and the absolute position at MWPC3 = M3.
	 \item theta\textunderscore out is the positive angle between the z-beam direction and the line between C and M3.
\end{enumerate}
The curvature radius $\rho$ is given by\footnote{for consistency checks the $\cos(\delta)$ term can be omitted, as it plays a minor role}:\\
$\rho$ = $\frac{\textrm{L\textunderscore eff}}{2\cdot\sin\left(\frac{theta\textunderscore in}{2} +\frac{theta\textunderscore out}{2}\right)\cdot\cos(\delta)}$\\
\\
With $\delta$: \\
$\delta = \arctan\left( \Bigg| \frac{\frac{cos(theta\textunderscore out)-\cos(theta\textunderscore in)}{\sin(theta\textunderscore out)+\sin(theta\textunderscore in)}+\tan(\alpha)}{1-\frac{cos(theta\textunderscore out)-\cos(theta\textunderscore in)}{\sin(theta\textunderscore out)+\sin(theta\textunderscore in)}\cdot\tan(\alpha)}\Bigg| \right)$ \\
The full derivation can be found in the appendix.\\
\\
The circular trajectory is then given by:\\
$\omega = 2*\Bigg|\arcsin\Big[\frac{BD}{2\cdot\rho}\Big] \Bigg|$\\
\\
with BD = length of the BD segment\\
\subsubsection{After GLAD up to the TOFW, the trajectory is a straight line}
\begin{center}
	\includegraphics[width=1.0\textwidth]{tracking_downstreamGLAD.png}
\end{center}
\textbf{The straight line trajectory from D to E is definded by:}\\
\\
\textbf{$\Rightarrow$ the output angle from GLAD theta\textunderscore out}\\
\textbf{$\Rightarrow$ one absolute position after GLAD in the laboratory frame M3}\\
\\
With this information the straight line trajectory lenght after GLAD can be measured. It starts at the exit point of GLAD D and follows the straigh line (characterized by the angle theta\textunderscore out and the absolute position at MWPC3) until the intersection with the TofW ( middle position of the ToFWall zToFW$=6660.2 mm$, tilted angle$=18^{\circ}$).\\
\\
Finally the pathlength in the (x,z) plane from the target position to the ToFW is given by:\\
P = AB + $\rho\cdot\omega$ +DE\\
\\
where:\\
A = (x,z) position at the target point \\
B = (x,z) position at the GLAD entry point\\
D = (x,z) position at the GLAD exit point\\
E = (x,z) position where the constructed trajectory line hits the ToFW\\
\\
The assumption for the "Kickplane" method is that the kickpoint for each event lies on the predefinded Kickplane, the symmetry axis line of the GLAD magnet. 

\subsection{The "Fit-Track" method}
For the "Fit-Track" method the assumption that the kickpoint C lies on the symmetry axis line of the GLAD magnet is rejected. Instead following algorithm is applied: \footnote{This algorithm is motivated from \url{https://www.blogs.uni-mainz.de/fb08-kernphysik/files/2018/09/PHDThesis_OlgaBertini.pdf}, section $3.4$  }\\
\begin{enumerate}
\item Extend the line of flight of the ion before the GLAD.
\item Draw a line from the point MW3 to C (as constructed with the "Kickplane" method).
\item Now sweep the straight line after the kickpoint, leaving the position MW3 unchanged but sweeping the intersection point along the inline beam.
\item For each sweeping step plot theta\textunderscore out versus (d1-d2) where d1 is the distance betweeen B and the point of intersection and d2 the distance between D and the intersection point accordingly.
\item 50 sweeping steps are performed.
\item Fit the final theta\textunderscore out versus (d1-d2) plot with linear least square fit. 
\item Find the intersection of the abscissa. The corresponding theta\textunderscore out value is now the corrected one which should be used for the calculation of the radius.
\end{enumerate}
\begin{center}
	\includegraphics[width=1.0\textwidth]{tracking_alg_ALADIN_s.png}
\end{center}
\begin{center}
	\includegraphics[width=0.5\textwidth]{theta_out_vs_d.png}
\end{center}


\subsection{The "Theta\textunderscore in correction" method}

Here the "Kickplane" method is used and subsequently the theta\textunderscore out is corrected by  theta\textunderscore in. That means:\\
\\
theta\textunderscore out\textunderscore corr = theta\textunderscore out - theta\textunderscore in.\\
\\
Consequentely the theta\textunderscore in dependence of $\rho$ vanishes(neglecting the $\cos(\delta)$ term):\\
\\

$\rho$ = $\frac{\textrm{L\textunderscore eff}}{2\cdot\sin\left(\frac{theta\textunderscore in}{2} +\frac{theta\textunderscore out\textunderscore corr}{2}\right)} = \frac{\textrm{L\textunderscore eff}}{2\cdot\sin\left(\frac{theta\textunderscore out}{2}\right)}$
\begin{center}
	\includegraphics[width=0.7\textwidth]{theta_out_corr_alg.png}
\end{center}

\subsection{Final method: "Advanced Fit-Track" method}
The same track finding algorithm as for the "Fit-Track" method is used with the only difference that the value for $theta\textunderscore in$ is calculated from the fit of $theta\textunderscore out$ vs. xMW3:\\
%insert plot here with fit theta_out vs xMW3
\\
The parameters of the linear fit are used for the calculation of $theta\textunderscore in$ :\\

$theta\textunderscore in = \alpha - a \cdot x\textunderscore MW3 - b$ \\
\\
With \textbf{a} being the slope and \textbf{b} the offset of the fit. This method prevents from adding up the errors from theta\textunderscore in measurement.\\
$\alpha$ corresponds to the calculated deflection angle depending on the GLAD current\footnote{for more info see: \url{http://web-docs.gsi.de/~land/glad/}}.

\section{Plots}
In this section all the plots for the various track finding algorithms are presented. For the calculation of the theta\textunderscore in angle MWPC1 and MWPC2 are used. Alternatively MWPC0 and MWPC2 could be used, to get a longer lever arm (work in progress ...).\\

\subsection{MWPC1 vs MWPC2 - x position}
\begin{figure}[!htbp]
\begin{subfigure}{.5\textwidth}
  \centering
  % include first image
  \includegraphics[width=.9\linewidth]{plot_imgs/mw2_mw1_get_centr.png}  
  \caption{"Kickplane-Method"}
  \label{fig:sub-first}
\end{subfigure}
\begin{subfigure}{.5\textwidth}
  \centering
  % include second image
  \includegraphics[width=.9\linewidth]{plot_imgs/mw2_mw1_corr.png} 
  \caption{"Theta \textunderscore in correction-Method"}
  \label{fig:sub-second}
\end{subfigure}
\begin{subfigure}{.5\textwidth}
  \centering
  % include second image
  \includegraphics[width=.9\linewidth]{plot_imgs/mw2_mw1_fit.png} 
  \caption{"Fit-Track-Method"}
  \label{fig:sub-second}
\end{subfigure}
\begin{subfigure}{.5\textwidth}
  \centering
  % include second image
  \includegraphics[width=.9\linewidth]{plot_imgs/mw2_mw1_last.png} 
  \caption{"Advanced Fit-Track-Method"}
  \label{fig:sub-second}
\end{subfigure}
\caption{MWPC1 vs MPWPC2 - x position for sweep runs 39-61.}
\label{fig:fig}
\end{figure}
\FloatBarrier
\clearpage
\subsection{MW0 vs MW2- x position }
\begin{figure}[!htbp]
\begin{subfigure}{.5\textwidth}
  \centering
  % include first image
  \includegraphics[width=.9\linewidth]{plot_imgs/mw2_mw0_get_centr.png}  
  \caption{"Kickplane-Method"}
  \label{fig:sub-first}
\end{subfigure}
\begin{subfigure}{.5\textwidth}
  \centering
  % include second image
  \includegraphics[width=.9\linewidth]{plot_imgs/mw2_mw0_corr.png} 
  \caption{"Theta \textunderscore in correction-Method"}
  \label{fig:sub-second}
\end{subfigure}
\begin{subfigure}{.5\textwidth}
  \centering
  % include second image
  \includegraphics[width=.9\linewidth]{plot_imgs/mw2_mw0_fit.png} 
  \caption{"Fit-Track-Method"}
  \label{fig:sub-second}
\end{subfigure}
\begin{subfigure}{.5\textwidth}
  \centering
  % include second image
  \includegraphics[width=.9\linewidth]{plot_imgs/mw2_mw0_last.png} 
  \caption{"Advanced Fit-Track-Method"}
  \label{fig:sub-second}
\end{subfigure}
\caption{MWPC2 vs MWPC0 - x position for sweep runs 39-61.}
\label{fig:fig}
\end{figure}
\FloatBarrier
\clearpage
\subsection{MW1 vs MW3 - x position}
\begin{figure}[!htbp]
\begin{subfigure}{.5\textwidth}
  \centering
  % include first image
  \includegraphics[width=.9\linewidth]{plot_imgs/mw3_mw1_get_centr.png}  
  \caption{"Kickplane-Method"}
  \label{fig:sub-first}
\end{subfigure}
\begin{subfigure}{.5\textwidth}
  \centering
  % include second image
  \includegraphics[width=.9\linewidth]{plot_imgs/mw3_mw1_corr.png} 
  \caption{"Theta \textunderscore in correction-Method"}
  \label{fig:sub-second}
\end{subfigure}
\begin{subfigure}{.5\textwidth}
  \centering
  % include second image
  \includegraphics[width=.9\linewidth]{plot_imgs/mw3_mw1_fit.png} 
  \caption{"Fit-Track-Method"}
  \label{fig:sub-second}
\end{subfigure}
\begin{subfigure}{.5\textwidth}
  \centering
  % include second image
  \includegraphics[width=.9\linewidth]{plot_imgs/mw3_mw1_last.png} 
  \caption{"Advanced Fit-Track-Method"}
  \label{fig:sub-second}
\end{subfigure}
\caption{MWPC3 vs MWPC1 - x position for sweep runs 39-61.}
\label{fig:fig}
\end{figure}
\FloatBarrier
\clearpage
\subsection{theta\textunderscore out vs MW3- x position }
\begin{figure}[!htbp]
\begin{subfigure}{.5\textwidth}
  \centering
  % include first image
  \includegraphics[width=.9\linewidth]{plot_imgs/theta_out_mw3_get_centr.png}  
  \caption{"Kickplane-Method"}
  \label{fig:sub-first}
\end{subfigure}
\begin{subfigure}{.5\textwidth}
  \centering
  % include second image
  \includegraphics[width=.9\linewidth]{plot_imgs/theta_out_mw3_corr.png} 
  \caption{"Theta \textunderscore in correction-Method"}
  \label{fig:sub-second}
\end{subfigure}
\begin{subfigure}{.5\textwidth}
  \centering
  % include second image
  \includegraphics[width=.9\linewidth]{plot_imgs/theta_out_mw3_fit.png} 
  \caption{"Fit-Track-Method"}
  \label{fig:sub-second}
\end{subfigure}
\begin{subfigure}{.5\textwidth}
  \centering
  % include second image
  \includegraphics[width=.9\linewidth]{plot_imgs/theta_out_mw3_last.png} 
  \caption{"Advanced Fit-Track-Method"}
  \label{fig:sub-second}
\end{subfigure}
\caption{Theta\textunderscore out vs MWPC3 x position for sweep runs 39-61.}
\label{fig:fig}
\end{figure}
\FloatBarrier
\clearpage
\subsection{theta\textunderscore in vs MW3 - x position}
\begin{figure}[!htbp]
\begin{subfigure}{.5\textwidth}
  \centering
  % include first image
  \includegraphics[width=.9\linewidth]{plot_imgs/theta_in_mw3_get_centr.png}  
  \caption{"Kickplane-Method"}
  \label{fig:sub-first}
\end{subfigure}
\begin{subfigure}{.5\textwidth}
  \centering
  % include second image
  \includegraphics[width=.9\linewidth]{plot_imgs/theta_in_mw3_corr.png} 
  \caption{"Theta \textunderscore in correction-Method"}
  \label{fig:sub-second}
\end{subfigure}
\begin{subfigure}{.5\textwidth}
  \centering
  % include second image
  \includegraphics[width=.9\linewidth]{plot_imgs/theta_in_mw3_fit.png} 
  \caption{"Fit-Track-Method"}
  \label{fig:sub-second}
\end{subfigure}
\begin{subfigure}{.5\textwidth}
  \centering
  % include second image
  \includegraphics[width=.9\linewidth]{plot_imgs/theta_in_mw3_last.png} 
  \caption{"Advanced Fit-Track-Method"}
  \label{fig:sub-second}
\end{subfigure}
\caption{Theta \textunderscore in vs MWPC3 x position for sweep runs 39-61.}
\label{fig:fig}
\end{figure}
\FloatBarrier
\clearpage
\subsection{theta\textunderscore in+theta\textunderscore out vs MW3- x position }
\begin{figure}[!htbp]
\begin{subfigure}{.5\textwidth}
  \centering
  % include first image
  \includegraphics[width=.9\linewidth]{plot_imgs/theta_out_theta_in_mw3_get_centr.png}  
  \caption{"Kickplane-Method"}
  \label{fig:sub-first}
\end{subfigure}
\begin{subfigure}{.5\textwidth}
  \centering
  % include second image
  \includegraphics[width=.9\linewidth]{plot_imgs/theta_out_theta_in_mw3_corr.png} 
  \caption{"Theta \textunderscore in correction-Method"}
  \label{fig:sub-second}
\end{subfigure}
\begin{subfigure}{.5\textwidth}
  \centering
  % include second image
  \includegraphics[width=.9\linewidth]{plot_imgs/theta_out_theta_in_mw3_fit.png} 
  \caption{"Fit-Track-Method"}
  \label{fig:sub-second}
\end{subfigure}
\begin{subfigure}{.5\textwidth}
  \centering
  % include second image
  \includegraphics[width=.9\linewidth]{plot_imgs/theta_out_theta_in_mw3_last.png} 
  \caption{"Advanced Fit-Track-Method"}
  \label{fig:sub-second}
\end{subfigure}
\caption{Theta \textunderscore out + theta \textunderscore in vs MWPC3 x position for sweep runs 39-61.}
\label{fig:fig}
\end{figure}
\FloatBarrier
\clearpage
\subsection{theta\textunderscore in vs Radius}
\begin{figure}[!htbp]
\begin{subfigure}{.5\textwidth}
  \centering
  % include first image
  \includegraphics[width=.9\linewidth]{plot_imgs/theta_in_rho_get_centr.png}  
  \caption{"Kickplane-Method"}
  \label{fig:sub-first}
\end{subfigure}
\begin{subfigure}{.5\textwidth}
  \centering
  % include second image
  \includegraphics[width=.9\linewidth]{plot_imgs/theta_in_rho_corr.png} 
  \caption{"Theta \textunderscore in correction-Method"}
  \label{fig:sub-second}
\end{subfigure}
\begin{subfigure}{.5\textwidth}
  \centering
  % include second image
  \includegraphics[width=.9\linewidth]{plot_imgs/theta_in_rho_fit.png} 
  \caption{"Fit-Track-Method"}
  \label{fig:sub-second}
\end{subfigure}
\begin{subfigure}{.5\textwidth}
  \centering
  % include second image
  \includegraphics[width=.9\linewidth]{plot_imgs/theta_in_rho_last.png} 
  \caption{"Advanced Fit-Track-Method"}
  \label{fig:sub-second}
\end{subfigure}
\caption{Theta \textunderscore in vs GLAD Radius for sweep runs 39-61.}
\label{fig:fig}
\end{figure}
\FloatBarrier
\clearpage
\subsection{theta\textunderscore out vs Radius}
\begin{figure}[!htbp]
\begin{subfigure}{.5\textwidth}
  \centering
  % include first image
  \includegraphics[width=.9\linewidth]{plot_imgs/theta_out_rho_get_centr.png}  
  \caption{"Kickplane-Method"}
  \label{fig:sub-first}
\end{subfigure}
\begin{subfigure}{.5\textwidth}
  \centering
  % include second image
  \includegraphics[width=.9\linewidth]{plot_imgs/theta_out_rho_corr.png} 
  \caption{"Theta \textunderscore in correction-Method"}
  \label{fig:sub-second}
\end{subfigure}
\begin{subfigure}{.5\textwidth}
  \centering
  % include second image
  \includegraphics[width=.9\linewidth]{plot_imgs/theta_out_rho_fit.png} 
  \caption{"Fit-Track-Method"}
  \label{fig:sub-second}
\end{subfigure}
\begin{subfigure}{.5\textwidth}
  \centering
  % include second image
  \includegraphics[width=.9\linewidth]{plot_imgs/theta_out_rho_last.png} 
  \caption{"Advanced Fit-Track-Method"}
  \label{fig:sub-second}
\end{subfigure}
\caption{Theta \textunderscore out vs GLAD Radius for sweep runs 39-61.}
\label{fig:fig}
\end{figure}
\FloatBarrier
\clearpage
\subsection{theta\textunderscore in vs theta\textunderscore out}
\begin{figure}[!htbp]
\begin{subfigure}{.5\textwidth}
  \centering
  % include first image
  \includegraphics[width=.9\linewidth]{plot_imgs/theta_in_theta_out_get_centr.png}  
  \caption{"Kickplane-Method"}
  \label{fig:sub-first}
\end{subfigure}
\begin{subfigure}{.5\textwidth}
  \centering
  % include second image
  \includegraphics[width=.9\linewidth]{plot_imgs/theta_in_theta_out_corr.png} 
  \caption{"Theta \textunderscore in correction-Method"}
  \label{fig:sub-second}
\end{subfigure}
\begin{subfigure}{.5\textwidth}
  \centering
  % include second image
  \includegraphics[width=.9\linewidth]{plot_imgs/theta_in_theta_out_fit.png} 
  \caption{"Fit-Track-Method"}
  \label{fig:sub-second}
\end{subfigure}
\begin{subfigure}{.5\textwidth}
  \centering
  % include second image
  \includegraphics[width=.9\linewidth]{plot_imgs/theta_in_theta_out_last.png} 
  \caption{"Advanced Fit-Track-Method"}
  \label{fig:sub-second}
\end{subfigure}
\caption{Theta \textunderscore in vs theta \textunderscore out for sweep runs 39-61.}
\label{fig:fig}
\end{figure}
\FloatBarrier
\clearpage
\subsection{MW3 vs Radius - x position}
\begin{figure}[!htbp]
\begin{subfigure}{.5\textwidth}
  \centering
  % include first image
  \includegraphics[width=.9\linewidth]{plot_imgs/mw3_rho_get_centr.png}  
  \caption{"Kickplane-Method"}
  \label{fig:sub-first}
\end{subfigure}
\begin{subfigure}{.5\textwidth}
  \centering
  % include second image
  \includegraphics[width=.9\linewidth]{plot_imgs/mw3_rho_corr.png} 
  \caption{"Theta \textunderscore in correction-Method"}
  \label{fig:sub-second}
\end{subfigure}
\begin{subfigure}{.5\textwidth}
  \centering
  % include second image
  \includegraphics[width=.9\linewidth]{plot_imgs/mw3_rho_fit.png} 
  \caption{"Fit-Track-Method"}
  \label{fig:sub-second}
\end{subfigure}
\begin{subfigure}{.5\textwidth}
  \centering
  % include second image
  \includegraphics[width=.9\linewidth]{plot_imgs/mw3_rho_last.png} 
  \caption{"Advanced Fit-Track-Method"}
  \label{fig:sub-second}
\end{subfigure}
\caption{MWPC3 x position vs GLAD Radius for sweep runs 39-61.}
\label{fig:fig}
\end{figure}
\FloatBarrier
\clearpage
\subsection{MW2 vs Radius - x position}
\begin{figure}[!htbp]
\begin{subfigure}{.5\textwidth}
  \centering
  % include first image
  \includegraphics[width=.9\linewidth]{plot_imgs/mw2_rho_get_centr.png}  
  \caption{"Kickplane-Method"}
  \label{fig:sub-first}
\end{subfigure}
\begin{subfigure}{.5\textwidth}
  \centering
  % include second image
  \includegraphics[width=.9\linewidth]{plot_imgs/mw2_rho_corr.png} 
  \caption{"Theta \textunderscore in correction-Method"}
  \label{fig:sub-second}
\end{subfigure}
\begin{subfigure}{.5\textwidth}
  \centering
  % include second image
  \includegraphics[width=.9\linewidth]{plot_imgs/mw2_rho_fit.png} 
  \caption{"Fit-Track-Method"}
  \label{fig:sub-second}
\end{subfigure}
\begin{subfigure}{.5\textwidth}
  \centering
  % include second image
  \includegraphics[width=.9\linewidth]{plot_imgs/mw2_rho_last.png} 
  \caption{"Advanced Fit-Track-Method"}
  \label{fig:sub-second}
\end{subfigure}
\caption{MWPC2 x position vs GLAD Radius for sweep runs 39-61.}
\label{fig:fig}
\end{figure}
\FloatBarrier
\clearpage

\section{Relative momentum resolution "Advanced Fit-Track-Method"}

The momentum resolution is calculated from the radius-calculation as $\rho \sim p$. From that follows:\\
\\
$\frac{\Delta p}{p} = \frac{\Delta\rho}{\rho}$
\\
For the evaluation of the resolutions for the various sweep runs the two dimensional plot "MWPC3.fX versus GLAD Radius" is projected on the abscissa. The resulting 1D plot is fitted with a gaussian. The mean value from the fit corresponds to $\rho$ and the $\sigma$ to $\Delta\rho$ respectively.\\
\\
\begin{tabular}{|c|c|c|c|}
\hline
Runnr. & $\overline{\rho}$ & $\sigma$ & rel. resolution \\
\hline
39     &         6.41710e+03    &     2.51358e+01    &     3.917e-03\\
40     &         6.40432e+03    &     1.39878e+01    &     2.184e-03\\
42     &         6.15987e+03    &     1.41755e+01    &     2.301e-03\\
44     &         5.79263e+03    &     1.28877e+01    &     2.224e-03\\
46     &         5.46248e+03    &     1.22759e+01    &     2.247e-03\\
48     &         5.16684e+03    &     1.03795e+01    &     2.009e-03\\
51     &         4.90002e+03    &     9.92633e+00    &     2.026e-03\\
53     &         6.66682e+03    &     1.70937e+01    &     2.564e-03\\
55     &         7.15478e+03    &     1.76967e+01    &     2.473e-03\\
57     &         7.71594e+03    &     2.15223e+01    &     2.789e-03\\
59     &         8.37120e+03    &     2.47619e+01    &     2.957e-03\\
61     &         9.13984e+03    &     2.95353e+01    &     3.231e-03\\
\hline

\end{tabular}
\\
\\
\newline
With mean relative resolution $\frac{\Delta\rho}{\rho}$ = 2.69e-03.



\end{document}